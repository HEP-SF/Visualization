% Copyright (C) 2018, The HSF Community White Paper authors, licence CC-BY-4.0.

% JHEP preprint template
\documentclass[12pt,a4paper]{article}
\usepackage{jheppub}

\usepackage{graphicx}
\usepackage{subcaption}
\usepackage{xspace}

\usepackage[utf8]{inputenc}


%%% BIBLIOGRAPHY SETTINGS %%%
\usepackage[style=numeric-comp,sorting=none]{biblatex}
%\addbibresource{../../roadmap/latex/cwp.bib}
%\addbibresource{../../roadmap/latex/cwp-chapters.bib} % HSF's one, for final submission
\addbibresource{cwp-chapters.bib} % local one, remove it for final submission
\addbibresource{cwp-wg-visualization-bib.bib}
%%%%%%


%% Comment out before final submission
\usepackage{lineno}  % for line numbering during review
\linenumbers

\abstract{In modern High Energy Physics (HEP) experiments visualization of experimental data has a key role in many activities and tasks
across the whole data chain: from detector development to monitoring, from event generation to reconstruction of physics objects,
from detector simulation to data analysis, and all the way to outreach and education.
In this chapter the definition, the status, and the evolution of data visualization for HEP experiments will be presented.
Suggestions for the upgrade of data visualization tools and techniques in current experiments will be outlined, along with guidelines for future experiments. This chapter expands on the summary content published in the HSF \emph{Roadmap} Community White Paper \cite{HSF-CWP-2017-01}.}

\begin{document}

\noindent
\begin{tabular*}{\linewidth}{lc@{\extracolsep{\fill}}r@{\extracolsep{0pt}}}
& & HSF-CWP-2017-15 \\
& & June 20, 2018 \\ % use \date or hardwire e.g. December 15, 2017
& & \\
\end{tabular*}
\vspace{2.0cm}

\title{HEP Software Foundation Community White Paper Working Group -- Visualization}

\author{HEP Software Foundation:}

\author[a,b]{Matthew Bellis}
\author[c,1]{Riccardo Maria Bianchi\note{Paper Editors}}%this same note will be shared with all the other authors with the [1] (i.e. Thomas McCauley)
\author[d]{Sebastien Binet}
\author[e]{Ciril Bohak}
\author[f]{Benjamin Couturier}
\author[g]{Hadrien Grasland}
\author[h]{Oliver Gutsche}
\author[i]{Sergey Linev}
\author[j]{Alex Martyniuk}
\author[k,1]{Thomas McCauley}
\author[l]{Edward Moyse}
\author[m]{Alja Mrak Tadel}
\author[n]{Mark Neubauer}
\author[o]{Jeremi Niedziela}
\author[p]{Leo Piilonen}
\author[q]{Jim Pivarski}
\author[r]{Martin Ritter}
\author[s]{Tai Sakuma}
\author[m]{Matevz Tadel}
\author[f]{Barthélémy von Haller}
\author[t]{Ilija Vukotic}
\author[j]{Ben Waugh}

\affiliation[a]{Siena College, Loudonville NY, USA}
\affiliation[b]{Cornell University, Ithaca NY, USA}
\affiliation[c]{University of Pittsburgh, Pittsburgh PA, USA}
\affiliation[d]{LSST, Alice, LPC}
\affiliation[e]{University of Ljubljana, Ljubljana, Slovenia}
\affiliation[f]{CERN, Geneva, Switzerland}
\affiliation[g]{LAL, Université Paris-Sud and CNRS/IN2P3, Orsay, France}
\affiliation[h]{FNAL, Batavia IL, USA}
\affiliation[i]{GSI Darmstadt, Germany}
\affiliation[j]{University College London, London, UK}
\affiliation[k]{University of Notre Dame, Notre Dame IN, USA}
\affiliation[l]{University of Massachusetts, Amherst MA, USA}
\affiliation[m]{University of California at San Diego, San Diego CA, USA}
\affiliation[n]{University of Illinois, IL, USA}
\affiliation[o]{Warsaw, Poland}
\affiliation[p]{Virginia Tech, VA, USA}
\affiliation[q]{Princeton University, Princeton PA, USA}
\affiliation[r]{LMU Munich, Munich, Germany}
\affiliation[s]{University of Bristol, Bristol, UK}
\affiliation[t]{University of Chicago, Chicago IL, USA}


\maketitle

\newpage


\hypertarget{scope}{%
\section{Scope}\label{scope}}

This paper will describe three kinds of data visualization used in High-Energy Physics (HEP): interactive visualization
of event data in applications known commonly as \emph{event displays}, statistical data visualization such as
histograms, and non-spatial data visualization such as networks and graphs.

Event displays are the main tool used to explore experimental data at the event level.
There are two main types of displays. The first type are those that are integrated in the experiments’ software frameworks,
which are usually able to access and visualize all experiment’s data at the cost of greater application complexity and lesser portability.
The second type of displays are those designed as cross-platform applications, lightweight and fast, often
delivering a simplified version or a subset of the event data. All event displays also show the detector geometry, at very different levels of detail depending on the application's use-case and targeted audience.

Beyond event displays, HEP also has statistical data visualizations such as histograms, allowing the data analyst to quickly
and with minimal effort characterize the data.
Unlike event displays, these visualizations are not strongly visually linked to the detector geometry,
and may aggregate data from multiple events. Data analysis tools and techniques used in HEP are
described in the HSF \textit{Data Analysis and Interpretation} Community White Paper~\cite{HSF-CWP-2017-05}.

Other types of visualizations are used in HEP to
visualize non-spatial data, like the graphs used to visually describe the structure of the detector description, that is 
the representation of all geometrical volumes that compose the sub-detectors and the infrastructure of a HEP experiment.
More details about the detector geometry can be found in the HSF \textit{Detector Simulation} Community White Paper~\cite{HSF-CWP-2017-07}

Other types of data visualization used in HEP experiments, such as visualization for slow control
or dashboards for data analytics, are considered out of scope and they will be not discussed in this chapter.

\hypertarget{current-landscape}{%
\section{Current landscape}\label{current-landscape}}

\hypertarget{event-displays}{%
\subsection{Event displays}\label{event-displays}}

Three key features characterize HEP event displays. The first is an \emph{event-based workflow}. Applications access
experimental data on an event-by-event basis,
visualizing the data collections belonging to a particular event. Data can be related to the
actual physics events ({\it e.g.}\ physics objects, like jets, tracks, and so forth) or to the experimental
conditions ({\it e.g.}\ different versions of the detector description, calibration data, and so on).

The second key feature is \emph{geometry visualization}.
The level of geometric detail displayed depends on the specific use-case, on the way the geometry information is stored and fetched ({\it e.g.}\ from a database as part of a software framework or from an external file), and on limitations of the application itself along with
considerations about speed, efficiency, and portability.

The third key feature is \emph{interactivity}. Applications offer different interfaces and tools for users to interact
with the visualization itself, select event data and set cuts on objects' properties.
In addition to the interactive usage, applications often store different settings to automate user’s actions.

In the following subsections several important aspects of data access, application development and distribution, and geometry description
and visualization, as they pertain to the current landscape of event displays, are discussed in more detail.



\hypertarget{data-access}{%
\subsubsection{Data access}\label{data-access}}

Access to event data comes either natively or via intermediate formats. In the former case direct access of native event formats
is only possible for an application integrated within the experimental software framework.

There are several advantages to having access to the experiment's framework, such as full access to the experimental data in its native format and to software tools, services, and databases. Through them, event display applications can make use of the full detector simulation geometry, of conditions data, and of all the framework APIs.

One disadvantage of this approach is that full support for the display application is often limited to those platforms
on which the framework itself is supported, limiting cross-platform distribution and support.
One way to mitigate is to distribute a light version of the framework along with the application; CMS Fireworks~\cite{CMSFireworks}
takes this approach. However, issues of platform support for the light framework and for the visualization application can still exist.
A further disadvantage to the full-framework (and even light-framework) approach is that one must also support various versions of
the data format along with the underlying framework API. In addition, users have to have knowledge of the framework in order to interactively
explore and visualize event-based data. Lastly, and for all the above mentioned reasons, often the user-interface to a
full-framework application is geared towards the expert.

Another approach to data access is via an intermediate format. Often the data needed for visualization is a subset of the
full information found in the native experimental format. In this case, one can extract what is needed from the framework
through the usage of dedicated exporting software tools and store them
into intermediate formats.

With the use of an intermediate data format (such as flavors of XML or JSON formats) the event display application
is potentially separate from the experimental software framework and therefore is not limited to platforms supported
by the framework. The data format and the application itself can be lightweight and the potential for widest distribution
of both is possible, even to the general public. As always there are drawbacks to any approach and in this case, with no
direct access to the experimental data, some information is necessarily not accessible.

Regardless of the approach to data access one must consider the use-case; what is useful or necessary in one use-case may not
be for another.


%%% MOVED TO SECTION "Suggested guidelines"
%But all these advantages come at a price, at least in the current landscape,
%as direct access to the experimental data is not possible. One has to export the required information from the experimental
%data into, for example, JSON, XML, or 3D-targeted file formats, and then render this information. However, in this process,
%one has distilled down from the experimental format and some information is lost or rather cannot be easily accessed.

%The access to experimental data is a problem in today’s experiments, and it is the most critical blocker when developing
%visualization tools for HEP. Developers have to choose from the beginning the data access pattern and the target of the
%application: either an integrated application which can visualize all data, but which can run only on specific platforms;
%or a cross-platform standalone tool, which can visualize only a subset of all the information. Moreover, the many
%differences in the access to data among the experiments, make very difficult to conceive common solutions and tools.

%Data, in fact, should be much more easily accessible to scientists, in a more transparent way and without the usage of
%complex software frameworks. That is an issue which is investigated and addressed in more details in the HSF
%"Data Organization, Management and Access" Community White Paper~\cite{HSF-CWP-2017-04}.

\hypertarget{application-development}{%
\subsubsection{Application development and distribution}\label{application-development}}

Currently the two most common ways of distributing event display applications are as a desktop application and as a web
application running in the browser. Each approach has its advantages and disadvantages which are further described in this section.

Mobile applications running on devices such as smartphones and virtual reality applications are less common in HEP. However,
they are a growing feature in the current landscape and many experiments are being done in exploring the possibilities of those emerging technologies. At the end of this section we describe briefly the current applications released, based on those technologies. Further developments will be described in
Section~\ref{modern-tech}.

{\bf Desktop applications} Many experiments have developed integrated event-display applications in C++, which is the main language used for developing
HEP software frameworks, on top of the OpenGL~\cite{OpenGL1992} application program interface (API).
The choice of the OpenGL API, compared to other APIs like Direct3D, resides in its cross-platform nature as
OpenGL is an open standard. The OpenGL consortium defines the API: the interface all the implementations have to
comply with. The actual implementation is provided by vendors, usually targeting a specific hardware. Many hardware and
software companies such as Intel and NVIDIA are part of the OpenGL consortium, which assure
the support and the lifetime of the OpenGL API.

Some HEP visualization applications use OpenGL calls directly through custom graphics engines. This is the most
robust approach as the developers can take full control over the OpenGL interface
and the project can be independent of other software libraries. Two examples of  HEP applications which followed this path are the
ATLAS Persint application~\cite{ATLASPersint2012} and the ROOT EVE toolkit~\cite{ROOTEVE2007}, which is used both by the CMS
Fireworks application~\cite{CMSFireworks} and the ALICE Event Visualisation Environment (AliEve)~\cite{alieve}.
One disadvantage of this approach is that personpower has to be assured to maintain the graphics engine.

Other applications use higher-level interface libraries as graphics engines. This has the advantage of delegating
a large part of the lower-level development work to external software packages, leaving the developers to concentrate
on the application development itself. One of the graphics libraries most used in HEP software has been so far
Open Inventor~\cite{OpenInventor1993}, used by the defunct CMS Iguana~\cite{CMSIguana} application, or its clone implementation
Coin (also known as Coin3D)~\cite{Coin3D}, which has been used by applications such the ATLAS VP1~\cite{ATLASVP12010},
the LHCb Panoramix~\cite{LHCbPanoramix} and the desktop version of the defunct CMS iSpy~\cite{CMSISpy}. Coin / Open Inventor was
chosen because of its integrability in C++ code, its performance, and its coding style. Moreover, the way Open Inventor handles
graphical volumes could be easily matched with the way geometry volumes are built for the experiments’ detector description.
Open Inventor organizes geometry volumes as a series of nodes in a tree-like structure in the same way as some
HEP experiments do. ATLAS, for instance, developed their geometry library “GeoModel”~\cite{ATLASGeoModel2004} based on
the same tree-like structure of nodes used by Open Inventor.

The drawback of this approach is the dependency on external projects, which could end up with a loss of functionality if
third-party library development and support are abandoned. Many scientific visualization applications, also in fields other than
HEP, faced this when the support of the Coin library was dropped by the company which lead its development~\cite{CoinEndOfLifeLetter}.
The result is the aging of libraries which after a while show incompatibilities with modern compilers and platforms. The time
spent by HEP developers to repair or to maintain those abandoned libraries results is time not spent on actual development
of the software applications themselves.

An additional approach to the development of event displays targeting the desktop is to create and distribute an application using Java. The ATLANTIS~\cite{ATLASAtlantis}
program and its derivative MINERVA~\cite{ATLASMinerva} which is used as an educational tool, both developed for the ATLAS experiment,
can be run either on the web or standalone on the desktop.


{\bf Web-based applications} Several experiments at the LHC, notably CMS~\cite{CMSISpyWebGL}, LHCb~\cite{LHCbOnline2014}, and ATLAS
~\cite{ATLASTada2016, ATLASTracer2015} have created web-based event displays using WebGL (Web Graphics Library)~\cite{WebGL2011}.
WebGL is a JavaScript API that conforms to OpenGL ES (a subset of the OpenGL API for embedded systems) conceived for
rendering interactive 3D and 2D graphics within any compatible web browser without the need of external plug-ins. With WebGL,
one can create high-quality graphics that were previously only available via bespoke desktop applications based on OpenGL and
graphical user interface toolkits such as Qt~\cite{QtFramework}.

Browser-based event displays have several distinct advantages: they are easy to distribute to the user, they can be prototyped quickly
, and the client is often much lighter-weight, as the need for building, packaging and distributing external libraries is greatly reduced.
There are also several mature and actively developed WebGL frameworks, such as three.js~\cite{ThreeJS}, that provide straight-forward
and simplified APIs for ease of development.

%Image: the image summarize the current landscape of event display applications in HEP. Many experiments developed
%full-framework desktop applications as well as light, web-based applications. As one can see from the plot, there
%are no examples so far of full-framework applications using web-based visualization graphics. A new approach in that
%direction is what it is suggested in this paper, in section [add ref].

{\bf Mobile and virtual reality applications} Nowadays mobile technology is more and more ubiquitous, people having access to a plethora of mobile devices: from tablets to
smartphones to ultrabooks. Those devices are used more and more as substitutes for desktop and laptop machines.


Mobile devices still do not have the computing power usually needed for HEP data analysis, where huge amount of experimental data
are retrieved and processed. In addition they usually run dedicated operating systems whose self-contained nature
makes their integration within the HEP workflow difficult, particularly for the statistical-based visualization used in
data analysis, described in Section~\ref{statistical-data-visualization}. Currently event visualization on mobile devices is only
possible in experiments which developed web-based tools. Only the visualization of events which have been already
extracted and reduced from the experiment's framework is possible.

There are at least a couple of examples of HEP applications developed for mobile platforms or that can be run on them. LHSee~\cite{LHSee}
was a mobile application which live streamed Atlantis events to a user’s phone and provided contextual information on ATLAS and the
events being displayed. The CMS iSpy WebGL-based application~\cite{CMSISpyWebGL} runs on mobile devices in the browser and users can interact
with the visualization with touch events. The Camelia application~\cite{CERNCamelia} and its successor TEV~\cite{CERNTEV} developed by the
CERN Media Lab using the Unity game engine, can be run on mobile devices as well.

Virtual reality (VR) describes the simulation of the user’s physical presence in a virtual environment. This simulation is typically
delivered via a HMD (Head Mounted Display) that provide visual and aural experience of the simulated environment. Rotational and
positional tracking of the user’s head and hand motion (when available) allow for interaction and motion in the virtual environment.

There are several ways to deliver VR to the user with varying levels of functionality, accessibility, and cost. They range from
applications running on a mobile phone viewed through simple headsets to the most realistic and immersive VR experiences provided
by the combination of advanced HMDs and desktop computers.

The simplest VR headset device is a Google Cardboard~\cite{GoogleCardboard} which allows the user to view content in stereo mode with
simple rotational tracking via a smart phone (using a mobile browser or a dedicated application) and a viewer. CMS iSpy WebGL running
in a mobile browser is one such application that can be used in this mode.

Currently, the most immersive and interactive VR environments are provided by the Oculus Rift and the HTC Vive, which combine the
computing resources of the desktop with sensors, controllers, and high-quality HMDs. Thanks to game engine’s abstraction of third
party VR libraries most HEP applications should be able to natively support both standard displays and all VR hardware.
Both ATLASrift~\cite{ATLASRift} and Belle II VR~\cite{BelleIIVR} support Oculus, HTC Vive, and 2D displays; CMS.VR [ref needed] for
Oculus and HTC. “More than ALICE” is an Augmented Reality application developed in Unity, allowing to superimpose detectors description or event
visualisation of the camera image of the ALICE detector or its paper model.

\hypertarget{geometry-description}{%
\subsubsection{Geometry description and visualization}\label{geometry-description}}

Geometry visualization provides important visual context for event displays and dedicated geometry displays are useful applications by themselves.
There are typically three levels of detail found in applications. The most detailed geometry is typically called the simulation geometry
and can include the sensitive elements of the detector as well as support structure. Less detailed is the so-called reconstruction
geometry, which describes the senstitive elements of the detector such as calorimeter elements and wire and strip chambers. It is this level of
detail that is typically found in event displays. The least-detailed geometry descriptions are those that are simplified versions of the
detector.

In framework-based applications the geometry information can come directly from the experiment's detector description and in
 many cases the hierarchical structure of the detector description is preserved and accessible. Standalone applications
typically use a geometry file with information exported from the software framework. An example hybrid solution is that of CMS and SketchUp~\cite{CMSSketchUp}.
The CMS detector description as written in XML is parsed using Ruby scripts and 3D models are built using the SketchUp program via its
Ruby API. SketchUp can then export to various standard 3D file formats. In this way detailed simulation geometry can be available in a standalone
application.


Currently, different geometry formats and libraries are used in HEP.
Some experiments use their own custom format, while others use the geometry tools provided by the ROOT. More recently, some attempts have been
done in order to build common formats and libraries for detector geometry, like DD4HEP,  adopted in conceptual design studies
for future high-energy colliders, including the CLICdp and FCC collaborations. In all cases detector
volumes are built from simpler geometrical entities: geometrical shapes like Tube, Cone, Box, and more complex variations of these
are combined in order to build the volumes of the experiment’s geometry. Geometry libraries and formats are described in more details
in the ??? HSF CWP White Paper.

The differences in geometry formats used by the different experiments, by detector simulation programs like Geant4~\cite{Geant4},
and by data analysis frameworks (like ROOT), typically require developers of visualization applications to write converters
between the different formats. In addition it is often not easy to use a visualization tool developed for one experiment with another one as
because current visualization tools are often tightly bound to the geometry library used by the experiment.

%CMS created 3D models of the CMS detector in SketchUp~\cite{CMSSketchUp}. Ruby scripts were used to parse the
%description of the CMS detector geometry written in XML and to build 3D models in SketchUp via its Ruby API.
%Figures produced based on these 3D models have been widely used: in many conference presentations, many PhD theses,
%technical design reports, a few notable journal publications, books, magazines, posters, brochures, websites, and other media.
%Further, the 3D models were imported in iSpy WebGL mentioned above.

%[ATLASED042016]
%[ATLASED072015]
%[OPEN-PHO-EXP-2015-013]
%[ALICE-EVENTDISPLAY-2017-007]
%{other images from other experiments ???}

%In the case of framework-integrated applications, the geometry shown is usually the actual geometry used in the detector description,
%taken directly from the experiment’s software framework and used for all kinds of tasks related to geometry (see Figure [ATLASED042016]).
%While in the case of  java-based standalone or web-based applications the geometry shown is usually a simplified version of the actual one
%(see figure [ATLASED072015]) or, merely, a fake geometry.

\hypertarget{statistical-data-visualization}{%
\subsection{Statistical data visualization}\label{statistical-data-visualization}}

Data visualization also means visualizing quantities and properties taken from a series of events, in order to extract statistical
meaning from them. An example of statistical data visualizations are histograms and scatter plots.

In HEP, like most other scientific disciplines, visualization of the data and of the final results plays a key role in the
analysis pipeline: the right projection of the data will suggest a new course of action; the results must be summarized in a
clear and concise way; multidimensional parameter spaces need to be visualized in an understandable fashion. The discussion
of how to properly display data is not new~\cite{Tufte1986}, but the tools are constantly evolving. Since its introduction 20 years ago,
ROOT~\cite{Root1997}, has become the most widely used package to make plots, graphics, and even event displays. It was developed at a
time when there were few alternatives to the community that did not have a significant financial cost and has performed admirably.

However, the landscape has changed and there are several existing tools, driven by non-HEP communities. This section will look at
some of the current alternatives and comment on what options might be available in the future and what our needs are. This section
will not address any visualization related to event displays, and instead focus on the making of plots and graphs, which 
spans a large parameter space of options on its own. For more details on HEP data analysis, please refer to the HSF \emph{Data Analysis and Interpretation} Community White Paper~\cite{HSF-CWP-2017-05}.

\hypertarget{stats-desktop}{%
\subsubsection{Desktop solutions}\label{stats-desktop}}

As it stands, ROOT is the most widely adopted plotting tool within the HEP and Nuclear Physics community.
It has even made some inroads to the astrophysics community and some small pockets within the financial community,
where some physicists migrated to. However, few other disciplines have adopted it. Still, it has many features beyond
the standard 1D/2D/3D histogram/graphing tools such as 2D and 3D shapes, widgets for building a GUI, a Javascript
implementation for web-based analysis~\cite{rootjs} and interacts with the Jupyter notebook. But to
access the plotting features, an analyst must install the entire ROOT package which includes file I/O, scientific
libraries, fitting routines, etc.\ and often the installation process is non-trivial.

Many current HEP analysts make wide use of the python programming language and the PyROOT libraries.
Python is also very popular outside the HEP community and so it is worth looking at non-ROOT options available to python users.
A recent (as of 2017) summary of the field was presented by Jake VanderPlas at PyCon 2017~\cite{VanderPlas2017}, a subset
of which will be presented here. It is emphasized that this is just a sampling and that the number of options available
is a function of time.

\begin{itemize}
\item Matplotlib~\cite{Hunter2007}. Released in 2003, this is the most mature plotting tool for python and is the standard
for most users. It can produce journal-quality graphics and there are some add-ons that can improve the default plotting
options~\cite{seaborn}. It does 1D, 2D, and 3D graphics with varying degrees of success, but does not
integrate with OpenGL libraries and so it can slow down when the number of data points gets very large.
It does produce most of the histograms found in HEP but some minimal, extra work must be done by the user
to make histograms with error bars. Plots are reactive in the sense that you can zoom in on different regions of the graph,
but you cannot do anything more significant with other mouseover commands (links, additional information, etc.).

\item The R programming language has several widely used graphics tools, both built-in or provided by external modules.
ggplot2~\cite{Wickham2009} and lattice~\cite{Sarkar2008} are particularly useful to visualize data in multidimensional parameter spaces.
ggplot2 is an implementation of The Grammar of Graphics~\cite{Wilkinson2005}. lattice is an implementation of Trellis Display~\cite{Trellis}.
Both packages are very popular outside the HEP community. A wide range of learning materials for both packages are available in
books, online courses, and other media. Both packages are well developed and matured. ggplot2 is now in the maintenance mode
and offers an official extension mechanism, with which users can easily develop new features. lattice has longer history.
In 2005, the year ggplot2 first appeared, lattice was already popular. In fact, figures made with lattice were shown in the
presentation in PHYSTAT05~\cite{phystat05} which introduced R to the
particle physics community. lattie is also well maintained. It is also easy for users to develop new features of lattice.
\end{itemize}

\hypertarget{stats-web}{%
\subsubsection{Web-based solutions}\label{stats-web}}

Web-based data visualization is also being rapidly developed. Very sophisticated toolkits now provide tools to build web-based
fully-responsive visualization of data on all types of devices. In addition, they also offer other features, specially useful
for HEP, like full in-browser LaTeX rendering (with MathJAX) and real-time visualization of streamed data. Being Javascript-based,
those libraries integrate with the overall ecosystem of web-based technologies, letting them use all the tools offered by other
web libraries. They are, overall, a good solution for data presentation, and can be combined with other tools such as Jupyter
in order to be used for data exploration. Some of the most used toolkits are listed below:

\begin{itemize}
\item D3 (Data Driven Documents)~\cite{D32011} is perhaps the first web-based visualization toolkit which has been widely
adopted as the de-facto base solution for building interactive data visualization for the web. The strong point of D3 is the
link of the data to the DOM entities and the possibility to work with SVG objects natively. D3 is also the foundation layer
upon which many higher level toolkits are built.

\item Bokeh~\cite{Bokeh2014}. This is a plotting utility from Continuum~\cite{continuum}, the company behind the
Anaconda distribution system and other python modules. It is designed with the web in mind and builds in a high degree of
interactivity into the plots, making it useful to share results publicly and for building dashboards. However, it works by
writing HTML, which makes it difficult to work with unless you use specific IDEs like a Jupyter notebook. Exporting a figure
for a journal article ({\it e.g.}\ png) is non-trivial as well, as that is not currently the primary use-case for bokeh.

\item Plotly~\cite{Plotly2015}. This is another web-oriented solution, similar to bokeh and based on the D3 library, where plots
can be hosted in Plotly’s cloud service or viewed in a Jupyter notebook. The plots are similarly very interactive and there
are ways to export figure images, but that is not the goal. Dashboards can be built with relative ease and plotly offers
libraries in R and Javascript, in addition to python. They offer both a free and enterprise business model.
\end{itemize}

The so-called notebooks are a rapidly evolving way of using web-based technology for both online and offline data analysis and
visualization, with access to local resources as well. After having started from a Mathematica-like notebook user interface
paradigm mixing server-side code snippet execution, structured text, and (mostly) static visualizations, the Jupyter community
is now exploring more interactive user interface paradigms, including in the area of visualization. The JupyterLab project is
exploring a more MATLAB-like IDE user experience inside of the web browser, with features such as multiple source editing tabs
and interactive python consoles. Its ipywidgets sub-project tries to make Jupyter more interactive by moving more visualization
work to client-side Javascript and introducing classic GUI widgets (sliders, checkboxes…) for interacting with the live visualization.
Belle II has started to use Jupyter notebooks to train new users in data analysis – the learning curve is much gentler than in
traditional terminal-based tutorials and the time to useful visualization of results is much faster.

In conclusion, there are mainly two classes of interactive plotting tools: data exploration tools and data presentation tools.
The first are those which you prepare and build your data visualization, while exploring and understanding your dataset;
the graphics part of ROOT, R, matplotlib and the Jupiter notebooks belong to this group. The data presentation tools, instead,
are those whose main goal is to present your final results in a convenient way; the web-based toolkits like D3 or Plotly
are part of this group.

\hypertarget{stats-issues}{%
\subsubsection{Issues}\label{stats-issues}}

\begin{itemize}
\item Separating data visualization from the data analysis:
The suite of statistical plotting tools in ROOT, Matplotlib, etc.\ are adequate for analysis, and their development is
very responsive to analysts’ needs. However, it is often hard to separate plot-making abilities from the data analysis framework.
As a consequence, if a physicist’s data can only be found on a particular server, the plot-generating code must also be located
there and are sometimes hard to bring to the physicist’s laptop screen. In the worst cases, graphics files (PNGs) must be copied
from the server to the laptop for viewing. This causes a high interaction latency, discouraging exploration. That’s why the
development of new tools should go towards a sharper separation between the computation on data and the interactive data visualization
routines, pushing the latter to the client side as much as possible.

\item Separating the plotting functions and content from the plotting style:
Another symptom of the tight coupling between data analysis infrastructure and plotting is that trivial changes to the plot— axis labels,
colors, and such— are so deeply buried in the analysis script that persistifying changes to them often requires a full recalculation
of the statistics. Changes to the final plot through the usage of on-display user interfaces, in fact, are overwritten and lost if a
plot is updated for other reasons (new version of data upstream, for example). Here, one could use inspiration from the increasing
separation of logic and presentation that is occurring in GUI toolkits (see {\it e.g.}\ use of CSS stylesheets in the GTK/GNOME environment).
A looser coupling between style and content, as well as a looser coupling between locality of computation and locality of rendering,
would benefit the physics community.
\end{itemize}

\hypertarget{non-spatial-visualization}{%
\subsection{Non-spatial visualization}\label{non-spatial-visualization}}

In HEP, there are data which are organized in a tree-like structure, and for which a graph or a network visualization is the best choice.

The Detector Description is an example of a source of such data: it describes all the pieces which compose a HEP experiment detector.
The different pieces of the Detector Description are interconnected through different relationships: geometrical volumes can be organized
in a parent-child relationship, or a property node can be shared among many volumes. The visualization of those data in a network helps
developers in the understanding and the debugging of the Detector Description, by visualizing the relationships among all the nodes and
their properties. The images below show an example of a graph visualizing the inner structure of a HEP detector description.

Image: graph visualizing the structure of the nodes describing a subset of Pixel detectors. Different colors identify different types
of node. In this example, the “boolean shape” nodes (in red) describing a subset of Pixel volumes are shown. [ATLASGeoModel2017]

Image: A graph visualizing the first layer of the nodes of the ATLAS Detector Description. The green nodes are the logical volumes used
by the top physical volumes (in blue), while the red nodes are name the subtree following them; the yellow and grey nodes are different
kinds of space transforms. Network or graphs are very effective ways of visualizing tree-like data, because they are able to show all the
nodes, their relationships and their properties in a proper way. Some degree of interactivity can let the scientists applying different
filters and layout, helping them to get rid of the clutter, to better understand and analyze the data [ATLASGeoModel2017].

Another example of HEP data that can benefit from a graph-based visualization is that one describing the execution chain of the jobs
used to filter and reconstruct the experimental data. Very recently HEP experiments [add refs] began to develop new parallel frameworks
to concurrently handle analysis or reconstruction jobs, to efficiently exploit the parallelism offered by the modern hardware.
The jobs are handled by a scheduler, which organize them according to their need input and output data. The outcome of the scheduler
is a directed acyclic graph (DAG). The visualization of that by mean of a graph helps the developer understanding and debugging the
reconstruction code and the experiment’s framework itself.

All those data are not space- nor time-dependent, and they are better visualized through a graph or a network. Graph-based visualization,
as well as graph-databases, are somehow new tools in the HEP landscape; but they can be very powerful tools to effectively visualize
non-spatial data which are by their nature organized with a network layout. Some experiments started to integrate them in their
toolkits [ATLASGeoModel2017, add other refs...], but we suggest the community to further explore those tools, to better explore the
possibility offered by graph-based solutions and exploit them for the HEP needs.

More examples from the other experiments, if any and if possible...

\hypertarget{suggested-guidelines}{%
\section{Suggested guidelines and future development}\label{suggested-guidelines}}

As a community, what we want to suggest here, is the design and the usage of common base visualization guidelines, to be able to share
knowledge and best practices among the communities, and to foster collaboration among the HEP experiments.

Visualization, as said many times already, has a key role in the lifecycle of a HEP experiment. And it takes input data from many
different sources and in many different formats.

The input data are often quite tightly bound to a specific experiment. And, because of that, many visualization tools are tightly
bound to their experiments as well. But, it is true that the output of a visualization application is not used by any other tool
within the experiment data chain. So, while the interaction with experiment data formats is highly experiment specific, there is a real possibility of having the final stages of the visualization pipeline shared between several experiments.

Let us take the example of the detector geometry. As said, there are very many different geometry libraries and formats in
use among the HEP experiments. However, geometry libraries are all different ways to describe and handle base geometrical
entities and combinations of them. At the end of the day, from a visualization point of view, the output of all geometry
libraries are mere descriptions of 3D volumes, which  could be abstracted from the underlying actual implementation.

A shape like a box, or a cone, or a tube, or some boolean combination of them, should be interpreted and handled the same by a
visualization tool in all experiment. Visualization applications should not be aware, and should not care, of how  the given 3D
volume has been created in the experiment’s code; they should only be able to correctly interpret the final information about it
and convert that to pixel values, to be displayed on the screen.

Of course, the usage of a common geometry library would solve many portability issues. But we know that this is not a viable
solution: experiments have very different needs, in terms of coding languages, available know-how, integration with other
software, portability, and so forth; and so they have to choose the geometry library which can fit the most of their needs.
Moreover, the experiments can decide to extend the geometry library they use, in order to add custom shapes or specific functionalities.
So, a common geometry library for all HEP needs cannot work.
But we could imagine common guidelines and, perhaps, base tools to visualize final geometry volumes. If we can imagine of
having base functions to visualize a community-defined Box shape, for example, we could imagine sharing the knowledge and the
workload among different experiments and communities. This does not mean that all experiments have to use a common definition
of a Box shape; but all experiments, whatever definition of a Box shape they use internally in their code for other reasons,
should be able to provide exporters from their internal definition to the community-agreed definition of a Box shape.
In that way, the know-how and the tools linked to visualization needs could be shared as well, and developed as a community.

We, as a community, still need to propose and design such common definitions and guidelines. This will
be addressed in the second phase of this Community action, following the completion of the Community White Paper.

For the moment, we observe that several experiments started to write exporters to translate geometry information to
standard formats used in the communities outside HEP, mainly in computer graphics and engineering. For example the Belle II
experiment has written exporters from Geant4 data to different formats, including VRML and FBX, which are two of the most common
formats used to store and share 3D graphics data. The Unity game-development engine~\cite{Unity3D}, in turn, can export the
FBX geometry to the glTF format~\cite{glTF}~\cite{glTF,SketchFabBelleII}, an emerging royalty-free specification for 3D objects and scenes,
for fast web distribution via the SketchFab community repository~\cite{SketchFab}. For the near future, we will foster similar
experiments within the community.

The same reasoning made for the geometry can be done for the event data, as well. Of course different experiments detect
different objects and measure different quantities. But there are many common entities, especially among experiments
within the same research field. For example, all experiments working on hadron colliders use the notion of particle track,
which is usually constructed translating the track measurements into space points or points and angles, and visualised as
a trail; or the notion of particle jet, usually visualised as a cone whose length is related to its energy and whose radius
is linked to the algorithm used for the jet reconstruction. Both those object are currently often handed and visualised
differently in different experiments, but we think that they could be the target of a common definition within the community.
If so, experiments could share best practices or snippets of code, if not complete base tools, to handle their visualisation.

START TEXT CUT FROM SECTION 2

However, portability and simplicity of usage are the strong points of mobile devices. More than as ``mobile'' devices, smartphones,
tablets and ultrabooks can be considered, as devices ``close to people''. As such, the usage of such devices
should be exploited more in the final steps of the visualization chain, where heavy batch data processing is not needed. For instance,
their usage should be leveraged for the production and visualization of event displays.
Ideally, a user should be able to easily retrieve interesting events from the experiment and interactively visualize them on all
kinds of devices.

That is why we strongly promote the usage of the server-client architecture described and supported in this paper in Section~\ref{client-server}
and the new data access patterns presented and supported in the ``Data Access and Management'' paper~\cite{HSF-CWP-2017-04}. This would open
up new possibilities for interactive visualization on mobile devices: it would let visualization clients running on mobile devices
connect to server tools running in the experiment's framework to easily and interactively retrieve the desired data.

It is worth noting that in other areas of science, for instance in astronomy, researchers have worked to facilitate data
access and to migrate to more standard data formats. This allowed for the possibility of having data visualization tools
on mobile devices, in addition to desktop and laptop machines. In addition, this helped the researchers easily accessing and visualizing
their data, but it also paid out in making science accessible by the public, having eased the development of programs used in
Outreach and Education activities and events. It is true that HEP data are usually much more complex than astronomy data, and so it
will be harder to achieve, but we think that an effort in simplifying the access to experimental data would be worth anyway.

Therefore, the leverage of the usage of mobile devices in HEP adds a strong point to the development and the support of common
client-server tools and data exchange formats [ref to paper] among HEP experiments in the near future.
END TEXT CUT FROM SECTION 2

\hypertarget{common-format}{%
\subsection{A common community-defined format}\label{common-format}}

In order to start sharing the knowledge and to start working on demonstrators to show and share best practices, we propose to
start defining a common format to exchange data among the experiments.

Initially, it does not have to be an actual format. We should start, instead, finding and listing common shared objects from geometry
and event data. After that, we should start converging on a shared definition of those objects, to build a common design toward
a data model to handle and serve them. The idea, in fact, is to enable usage of this common format to visualize data from the
different experiments with the same shared best practices, if not the same tools.

We think that community-developed common formats and tools should also be extendable, to let the experiments add their own
custom content and objects. As an example, calorimeter cells can be of very different shapes, and an experiment might need to
add its own custom shapes to the common format to visualize them properly. Thus, in addition to the part handling the common objects,
there should be a part of the format targeted at storing extended custom content, specific to a given experiment. For such
custom content, experiments will have to develop custom visualization tools as well; but they could build them upon the foundation
of the community-driven part.

Some experiments in that direction have been performed within the community in the past years, already. For example,
the ALICE experiment made use of the mini ``Visualization Summary Data'' (VSD) set of classes, contained in the ROOT Event
Visualization Environment (EVE), to make ALICE data visualization decoupled from the AliROOT experiment’s framework.

\hypertarget{serving-data}{%
\subsection{Serving the geometry and event data through services}\label{serving-data}}

Once a common format for shared objects is defined, we believe that design and development of online services to query and
serve the geometry data would be a very useful addition, compared to what the HEP experiments currently feature.
The main driving force here is the realization that detector description should be much more accessible than it is today. For many experiments,
accessing the detector description means starting and running at least parts of the experiment’s framework. Accessing a specific
geometry version or the latest one is of course critical for reconstruction and simulation; but the geometry data needed for
event visualization can be simpler. Even when showing the actual geometry of the experiment, accessing the latest alignment
constants is not crucial for visualization purpose, because small differences in the geometry are not visible in an event display.
So, we think that serving a “frozen” version of the experiment’s geometry would be enough, and that a simpler way to retrieve
it should be designed, to ease data access for visualization; for example, through an online service which can be queried by
visualization applications.

It would be desirable for the new mechanism to have a search/filter functionality too, to let client applications query
for a specific subset of information; and a way to select the level of details, to set the desired accuracy and complexity
of the retrieved geometry.

The same could be envisaged for event data, even though that is a more complicated task, involving many different layers and services:
very often event data are stored on the Grid and very often they need to be processed in order to be usable for event displays.
But, as discussed earlier, experimental data definitely should be more accessible for visualization. Thus, in collaboration with
the Data Access and Management WG, an API or a service to get streamed event data will be designed.

In addition, simulation data description for visualization could be handled in the same way, by the usage of converters from
generators/simulation applications.

After a first phase of development and stand-alone testing, the streamed data could be used by the current visualization tools
as well, as first step of their modernization and towards the usage of common community-developed techniques and shared best practices.

\hypertarget{client-server}{%
\subsection{Client-server architecture for geometry and event data visualization}\label{client-server}}

After common data formats and a mechanisms to serve it are designed together with a set of exporters required to translate the
experiments’ data to the common format, we are proposing to build a client-server architecture, upon which next generation visualization applications can be built.
The idea behind that is that if we can send commands from the client to the server, and get the answer back in the data stream,
then we will be able to interact with the experiment’s framework as well, in addition of using common visualization applications
to visualize the common objects. In this way, we will achieve to develop a modular architecture where HEP experiments could share
the design, the development and the maintenance of common visualization tools, while maintaining a certain degree of freedom to
add custom content and objects and to interact with their own framework to retrieve specific content.

\hypertarget{modern-tech}{%
\subsection{Exploring modern technologies}\label{modern-tech}}

\hypertarget{graphic-engines}{%
\subsubsection{Graphics engines}\label{graphic-engines}}

As briefly told earlier, so far direct OpenGL or old graphics libraries have been used in HEP visualization applications. Nowadays,
another type of graphics library is rapidly evolving, those embedded in the so-called game engines. Those are software frameworks
targeted at the gaming industry, and they feature very efficient, optimized and modern 3D graphics.
The integration with existing code is not easy, because they are usually meant to be used as development environments, and not as
embedded libraries like those used in HEP so far. So they would probably require some major changes in the usual software architecture
used in HEP. But they offer very optimized graphics and modern features, like tools for Virtual Reality, which could be exploited in our tools.

Some HEP experiments have recently started to successfully use them to build visualization applications and event displays,
like Belle II~\cite{BelleIIVR} and the Total Event Visualiser (TEV) of the CERN Media Lab~\cite{CERNTEV}, which used the Unity
game engine~\cite{Unity3D}, and ATLAS which used the Unreal Engine~\cite{EpicUnreal} for its virtual reality application
ATLASrift~\cite{ATLASRift}.

Those game engines are the most popular ones on the market, and they are free for educational and non-commercial projects.
Moreover, Unreal Engine is fully open source. It supports two modes of development (C++ and Blueprints) that can be used
interchangeably even in the same project. It produces extremely performant executables for basically all platforms
(Windows, MacOS, Linux, iOS, Android, Web, all VR platforms). All parts of development cycle are fast even for a novice,
thanks to powerful tools implemented as plugins. It has a huge developers community and is very fast in supporting the
latest technologies {\it e.g.}\ it already supports Vulkan - cross-platform 3D and compute API. Unreal Engine is used by the ATLASrift.
As said, Unity has been used successfully by the Belle II experiment. The Unity development platform~\cite{Unity3D} is very
intuitive for novices as well as experts and provides rapid turnaround during the development cycle: the project can be
executed immediately (using the platform’s ``Play'' button) without having to compile and link an executable (which takes
longer and is done when the user’s changes have stabilized) for the same platforms as listed above for the Unreal Engine.
All VR devices are supported as targets and VR-animation performance is reported to be significantly better on low-end devices.
(Notably, the emerging WebVR standard~\cite{WebVR} (recently replaced by WebXR \cite{WebXR}) is not yet supported for ``VR in the browser''. It is possible in Unity to create a
WebGL~\cite{WebGL2011} executable of a VR app; however, no browser will run this app successfully because of its need to load the
VR-hardware’s object library, which violates security restrictions in the browser.) Presently, user code is written in either
C\# or an adaptation of Javascript. These languages are somewhat foreign to the majority of HEP-trained programmers who are more comfortable with
C++; however, this accounts in part for the broader use of Unity than Unreal.

Another game engine that is gaining a lot of attention in recent years is an open source engine Godot~\cite{Godot}. 
While it is still not quite at the same level as Unity or Unreal Engine, it allows the deployment to similar platforms as Unity and Unreal Engine but is very lightweight. 
It offers the support for 2D and 3D graphics and for multiple programming languages {\it e.g.}\ GDScript (a Python-like scripting language), C\# 7.0 (by using Mono), and C++. 
It also offers visual scripting using blocks and connections and support for additional languages with community-provided support for Python, Nim, D and other languages.

As a community, we would like to explore further the features those modern game engines can offer. Also, we would like to take a
look at possible usage patterns in the context and within the workflow of HEP visualization.

Another new entry in the 3D graphics engines landscape is Qt3D~\cite{Qt3d},  the new 3D engine of Qt. The key feature of
Qt3D is that it is natively integrated with the Qt framework, which eliminates a layer which was needed until now: a glue
package to connect the Qt GUI with the window showing the 3D content. By eliminating that, we could simplify the architecture
of our visualization tools and lower the maintenance work. Qt3D is still in development, but it shows an initial set of
features which are worth a further consideration of the new toolkit. We plan to take a look at its development in the near future,
to see if it can satisfy the HEP requirements. Also, being open source, we could consider contributing to the Qt3D software
project as a community, by providing the pieces we need for our applications.

\hypertarget{web-based}{%
\subsubsection{Web-based applications}\label{web-based}}

Web-based graphics have traditionally been considered not powerful enough to handle the thousands of volumes that can be
shown in a HEP event displays, for example, when visualizing hits in a very busy event. But the technology has rapidly
evolved with strong support as well. Web-based graphics can now visualize very complex, busy events.

Therefore, there is a strong interest in the community in supporting the usage and the development of tools around
these technologies. In particular, we are currently interested in supporting JSROOT~\cite{rootjs}, which could be used as
underlying layer for event data visualization as well, three.js~\cite{ThreeJS} and WebGL~\cite{WebGL2011}, which have been
used successfully by different experiments [add refs] to visualize geometry and event data too. The glTF~\cite{glTF} 3D model
of the Belle II detector~\cite{BelleII}, with tens of thousands of elements, can be loaded, viewed and manipulated in a web
browser – even on a smartphone – very effectively~\cite{SketchFabBelleII}.

\hypertarget{vr}{%
\subsubsection{Virtual reality}\label{vr}}

Virtual reality (VR) describes the simulation of the user’s physical presence in a virtual environment. This simulation
is typically delivered via a Head Mounted Display (HMD) that provides visual and aural experience of the simulated environment.
Rotational and positional tracking of the user’s head and hand motion (when available) allow for interaction and navigation
in the virtual environment.

There are several ways to deliver VR to the user with varying levels of functionality, accessibility, and cost.
They range from applications running on a mobile phone viewed through simple headsets to the most realistic and
immersive VR experiences provided by the combination of advanced HMDs and desktop computers.

The simplest and most inexpensive way to deliver VR is via the browser on a mobile phone and viewed through a
Google Cardboard headset. The Cardboard viewer can itself be literally made from cardboard. With device orientation
controls either in a native application or using the HTML5 device orientation API in the browser one has rotational
tracking ({\it i.e.}\ 3 degrees-of-freedom). No hand controller is used in the Cardboard but for native applications a click event
is available via a magnet attached to the Cardboard viewer.

The Google Daydream is the next iteration in VR for Google. Content is still delivered by a mobile phone but a
Bluetooth-connected hand controller (with only rotational tracking) is available. Stand-alone HTC Vive and Lenovo
headsets for the Daydream environment will be available in the near future.

Currently, the most immersive and interactive VR environments are provided by the Oculus Rift and the HTC Vive,
which combine the computing resources of the desktop with sensors, controllers, and high-quality HMDs. Thanks to
game engine’s abstraction of third party VR libraries most HEP applications should be able to natively support both
standard displays and all VR hardware. Both ATLASrift~\cite{ATLASRift} and Belle II VR~\cite{BelleIIVR} support Oculus, HTC Vive,
and 2D displays; CMS.VR (not released yet) for Oculus and HTC.

Development in the mobile browser for VR can be done using a WebGL library such as three.js~\cite{ThreeJS} and using the
HTML device orientation control API. The viewport is split into two views for each eye each with separate cameras separated
by an appropriate distance to create a stereoscopic effect ({\it e.g.}\ iSpy WebGL has a stereo mode for Google Cardboard)
The developing WebVR specification~\cite{WebVR} provides interfaces to VR hardware via the browser. A powerful framework for
development of VR applications for various devices using the browser is A-Frame~\cite{AFrame}. ({\it e.g.}\ CMS A-Frame prototype)

Game engines, described in the previous section [add ref],  provide powerful integrated development environments for creation
of VR applications for multiple devices.

\hypertarget{multi-user}{%
\subsubsection{Multi-user applications}\label{multi-user}}

Nowadays multi-users technology is used in many applications: for example in GoogleDocs, where many users can simultaneously
interact with the same document. What we would like to provide, is a multi-user support for visualization, to let several users
explore and interact with events at the same time. Beside being a useful feature for expert users, it could be important for
Outreach and Education activities, where people or students could interact together with an event display.
Game engines offer multi-users support natively, thus we could starting explore their usage.
An example of such collaborative features in a 3D environment is integrated in Med3D visualization framework~\cite{Bohak2017}.

\hypertarget{sharing-knowledge}{%
\section{Sharing knowledge and fostering collaboration}\label{sharing-knowledge}}

During the kickstarter meetings and the different workshops organized to start and develop the present Community White Paper,
the whole HSF Visualization Working Group has agreed on the importance of sharing the knowledge among the whole HEP community,
as well as the best practices and the know-how. Too often, in fact, solutions and tools developed for one HEP experiment are
not sufficiently advertised to the rest of the HEP Visualization community, with the result that the community base knowledge
is fragmented and not efficiently exploited.

The focus of this WG, in fact, is not limited to the preparation of this white paper. Instead, a longer term collaboration
among the experiments is foreseen, in order to collaborate on common visualization projects.

To foster collaboration and sharing, we agreed on different points, which are listed below.

\hypertarget{workshop}{%
\subsection{Yearly workshop}\label{workshop}}

On March 2017 the first HSF Visualization Workshop has been organized, at CERN, to let all the experts from the different
show their work and share their solutions. Also, external experts from industry have been invited, to present the latest
advancements in the field and best practices.

To maximize the participation within the HEP community, the workshop has been run over three afternoons, to let the
colleagues from US to connect.

It was the first topical workshop focused on HEP Visualization since many years. And it has been a success. Many HEP
experiments and communities showed their latest developments. Given the high number of presentations, a second one-afternoon
workshop has been organized has a follow-up, to let the remaining communities to present their work.

The Working Group agreed on the importance of meeting to share findings, knowledge and solutions; and it was decided to
try to organize a topical workshop on HEP visualization once per year.

An important point was raised while organizing the first Workshop: other scientific fields have visualization and
graphics needs similar to ours --- let us think, for example, at Geophysics. In the future workshops we will try to
have presentations from other communities as well, in order to try to foster friendly and fruitful collaborations,
which could benefit the whole scientific community.

\hypertarget{repo}{%
\subsection{Code repository}\label{repo}}

The Working Group also agreed on the importance of fostering collaborative work.
In order to start that, a new code repository has been created within the HSF GitHub repository~\cite{HSFVizRepo}.
It is the space where members of the WG can share their work-in-progress studies and their solutions, and where
community-driven projects will be stored.

\hypertarget{roadmap}{%
\section{Roadmap}\label{roadmap}}

\hypertarget{one-year}{%
\subsection{One year}\label{one-year}}

In the first year the Visualization Working Group (WG) will work on defining R{\&}D projects, based on the key
points and ideas discussed in this community white paper.

The main goal will be developing techniques and tools which let visualization applications and event displays be
less dependent on specific experiments’ software frameworks, leveraging the usage of common packages and common data formats.

In a first phase, the community will identify common objects and will agree on common definitions, as described in
section [add ref]. Then, a common data exchange format --- either based on custom data formats or, if possible, on
open standards --- will be designed by the community. After that, exporters and interface packages would be designed
as bridges from the experiments’ frameworks -- which are needed to access data at a high level of detail -- to the common packages.

\hypertarget{three-year}{%
\subsection{Three years: ATLAS and CMS Computing TDRs}\label{three-year}}

In the second and third year the Visualization WG will work on designing and building demonstrators to show the feasibility of
the community-driven best practices and tools. The goal will be to get a final design of those tools, to be included in the development
plans of the different experiments. Moreover, the WG will work towards a more convenient access to geometry and event data.
In collaboration with the Data Access and Management WGs, an API or a service to get streamed event data would be designed.

\hypertarget{five-year}{%
\subsection{Five years: Towards HL-LHC}\label{five-year}}

In the fourth and fifth year, the focus will be on developing the actual community-driven tools, to be used by the experiments
for their visualization needs in production.

The goal will be the usage of the community-developed tools within the experiments’ visualization applications; and
perhaps the usage of a simplified data access, but that depends on the actual feasibility, which will be established
after an initial study.

\hypertarget{conclusions}{%
\section{Conclusions}\label{conclusions}}

Modern and modular visualization tools, which will feature simplified data access and retrieval as well, would leverage
the accessibility, letting end users exploit all the possibilities offered by modern visualization solutions, without the need
of running them on specific platforms, running them within the experiments’ software frameworks, or being bound to specific solutions.
And a better experience will reflect to a better usage, which will positively affect the usage of such visualization tools for
detector development and simulation of new experiments, as well as the data analysis and the upgrade studies of the current ones.

In the end, common community-driven tools will let users of all experiments use the latest and best tools, while sharing the development,
the maintenance and the workload among all the experiments.
 
\hypertarget{acknowledgements}{%
\section{Acknowledgements}\label{acknowledgements}}


???

%\bibliography{cwp-wg-visualization}%,cwp-chapters}
%%\bibliography{cwp-chapters}
%\bibliographystyle{IEEEtran}

%\begin{thebibliography}{999}b
%
%\bibitem{hsf-cwp-roadmap} ``A Roadmap for HEP Software and Computing R{\&}D for the 2020s'', HSF-CWP-2017-001, arXiv:1712.06982
%
%%\bibitem{DataAnalysisCWP2017} HSF Data Analysis and Interpretation Community White Paper (??? link to the final paper)
%
%\bibitem{hsf-cwp-simulation} HSF Simulation CWP
%
%\bibitem{CMSFireworks} L.A.T. Bauerdick et al., "Event display for the visualization of CMS events", J.Phys.Conf.Ser. 331 (2011) 072039
%  {\tt https://twiki.cern.ch/twiki/bin/view/CMSPublic/WorkBookFireworks} 
%
%\bibitem{hsf-cwp-data} HSF Data Organization, Management, and Access CWP
%
%\bibitem{hsf-cwp-analysis} HSF Analysis CWP
%
%\bibitem{OpenGL1992} {\tt https://www.opengl.org/about}
%
%\bibitem{ATLASPersint2012} ATL-SOFT-PUB-2012-001 {\tt http://cds.cern.ch/record/1501131},
%{\tt https://twiki.cern.ch/twiki/bin/viewauth/Atlas/PersintWiki}
%
%\bibitem{ROOTEVE2007} {\tt https://root.cern.ch/eve}
%
%\bibitem{alieve} AliEve ref needed
%
%\bibitem{OpenInventor1993} J. Wernecke, "The  Inventor Mentor: Programming Object-Oriented 3D Graphics with Open Inventor, Release 2", 1993, Addison-Wesley
%
%\bibitem{CMSIguana} ref to Iguana
%
%\bibitem{Coin3D} https://bitbucket.org/Coin3D/coin/wiki/Home
%
%\bibitem{ATLASVP12010} T. Kittelmann, V. Tsulaia, J. Boudreau and E. Moyse, "The Virtual Point 1 event display for the ATLAS experiment", Journal of Physics: Conference Series, 2010, vol. 219 n. 3 ( http://iopscience.iop.org/1742-6596/219/3/032012) -
%{\tt https://atlas-vp1.web.cern.ch/atlas-vp1}
%
%\bibitem{LHCbPanoramix} {\tt http://lhcb-comp.web.cern.ch/lhcb-comp/frameworks/Visualization/}
%
%\bibitem{CMSISpy} G. Alverson et al. "iSpy: A powerful and lightweight event display", J.Phys.Conf.Ser. 396 (2012) 022002
%
%\bibitem{ATLASGeoModel2004} J. Boudreau and V. Tsulaia, "The GeoModel Toolkit for Detector Description", CHEP 2004, {\tt https://cds.cern.ch/record/865601/}
%
%\bibitem{ATLASAtlantis} {\tt http://www.cern.ch/atlantis}
%
%\bibitem{ATLASMinerva} {\tt http://cern.ch/atlas-minerva}
%
%\bibitem{CMSISpyWebGL} T. McCauley "A browser-based event display for the CMS experiment at the LHC using WebGL", J.Phys.Conf.Ser. 898 (2017) no.7, 072030
%  ({\tt http://cern.ch/ispy-webgl})
%
%\bibitem{LHCbOnline2014} {\tt https://lhcb-public.web.cern.ch/lhcb-public/lbevent2/lbevent/}
%
%\bibitem{ATLASTada2016} G. Sabato et al., CHEP 2016, "ATLAS Fast Physics Monitoring: TADA",  {\tt https://cds.cern.ch/record/2244283}
%
%\bibitem{ATLASTracer2015} {\tt https://atlas-tracer.web.cern.ch/}
%
%\bibitem{WebGL2011} {\tt https://www.khronos.org/webgl/}
%
%\bibitem{ThreeJSXXXX} {\tt https://threejs.org/}
%
%\bibitem{LHSee} {\tt https://www2.physics.ox.ac.uk/about-us/outreach/public/lhsee}
%
%\bibitem{CERNCamelia} {\tt http://medialab.web.cern.ch/content/camelia}
%
%\bibitem{CERNTEV} {\tt https://gitlab.cern.ch/CERNMediaLab/TEV} ???
%
%\bibitem{GoogleCardboard} {\tt https://vr.google.com/cardboard/}
%
%\bibitem{ATLASRift} {\tt https://atlasrift.web.cern.ch/}
%
%\bibitem{BelleIIVR} {\tt http://www1.phys.vt.edu/~piilonen/VR/}
%
%\bibitem{CMSSketchUp} T. Sakuma, T. McCauley, "Detector and Event Visualization with SketchUp at the CMS Experiment", J.Phys.Conf.Ser. 513 (2014) 022032
%
%\bibitem{Root1997} R. Brun and F. Rademakers, Nuclear Instruments and Methods A, Volume 389 (1997) 81-86, (http://dx.doi.org/10.1016/S0168-9002(97)00048-X)
%
%\bibitem{Geant4} J Allison et al., "Recent developments in Geant4", Nuclear Instruments and Methods in Physics Research A 835 (2016) 186-22
%  {\tt http://geant4.cern.ch/}
%
%\bibitem{Tufte1986} E. R. Tufte, "The Visual Display of Quantitative Information" (1986) 0-9613921-0-X, Graphics Press, Cheshire, CT, USA.
%
%\bibitem{rootjs} {\tt https://root.cern.ch/js}
%
%\bibitem{VanderPlas2017} {\tt https://speakerdeck.com/jakevdp/pythons-visualization-landscape-pycon-2017}
%
%\bibitem{Hunter2007} J.D. Hunter, "Matplotlib: A 2D graphics environment", Computing In Science \& Engineering, Volume 9, Number 3, 90-95, 2007 (http://dx.doi.org/10.1109/MCSE.2007.55)
%
%\bibitem{seaborn} {\tt https://seaborn.pydata.org}
%
%\bibitem{Wickham2009} H. Wickham. ggplot2: Elegant Graphics for Data Analysis. Springer-Verlag New York, 2009
%
%\bibitem{Sarkar2008} D. Sarkar. Lattice: Multivariate Data Visualization with R. Springer, New York. ISBN 978-0-387-75968-5
%
%\bibitem{Wilkinson2005} L. Wilkinson. The Grammar of Graphics, Springer, ISBN 978-0387245447
%
%\bibitem{Trellis} http://ect.bell-labs.com/sl/project/trellis/software.writing.html
% R. A. Becker and W. S. Cleveland. S-PLUS Trellis Graphics User's Manual, MathSoft 1996
%
%\bibitem{phystat05} {\tt http://www.physics.ox.ac.uk/phystat05/Talks/PhyStat05-paterno.pdf}
%
%\bibitem{D32011} M. Bostock et al., "D3: Data-Driven Documents", IEEE Trans. Visualization \& Comp. Graphics (Proc. InfoVis) (2011), (http://vis.stanford.edu/papers/d3)
%
%\bibitem{Bokeh2014} {\tt http://www.bokeh.pydata.org}
%
%\bibitem{continuum} {\tt https://www.continuum.io/}
%
%\bibitem{Plotly2015} {\tt https://plot.ly}
%
%\bibitem{ATLASED042016} {\tt https://cds.cern.ch/record/2148236}
%
%\bibitem{ATLASED072015} FIX ME %FIX https://twiki.cern.ch/twiki/pub/AtlasPublic/EventDisplayRun2Collisions/JiveXML_271298_403602858-RZ-LegoPlot-EventInfo-RZ-YX-2015-08-06-15-01-42.png
%
%\bibitem{ATLASEventDisplays2016} https://indico.cern.ch/event/505613/contributions/2228333/
%
%\bibitem{ATLASGeoModel2017} R.M. Bianchi and I. Vukotic, ACAT 2017, "A scalable new mechanism to store and serve the ATLAS detector description through a REST web API", https://indico.cern.ch/event/567550/contributions/2628864/
%
%\bibitem{EpicUnreal} {\tt https://www.unrealengine.com}
%
%\bibitem{HSFVis2017} ``HSF Visualization Workshop'' ({\tt https://indico.cern.ch/event/617054/})
%
%\bibitem{HSFVizRepo} {\tt https://github.com/HEP-SF/Visualization}
%
%\bibitem{LHCbEventDisplays2017} B. Couturier, ``Event Displays in LHCb" ({\tt https://indico.cern.ch/event/617054/contributions/2529736/})
%
%\bibitem{Tukey1962} J. W. Tukey, ``The Future of Data Analysis", The Annals of Mathematical Statistics Volume 33, Number 1, 1-67 (1962) ({\tt http://www.jstor.org/stable/2237638})
%
%\bibitem{glTF} {\tt https://www.khronos.org/gltf/}
%
%\bibitem{SketchFab} {\tt https://sketchfab.com}
%
%\bibitem{SketchFabBelleII} {\tt https://sketchfab.com/models/6c7aa0c71dc64079b08a0dddfa756ef3}
%
%\bibitem{Unity3D} {\tt https://unity3d.com}
%
%\bibitem{WebVR} {\tt https://w3c.github.io/webvr/}
%
%\bibitem{AFrame} {\tt https://aframe.io/}
%
%\bibitem{BelleII} {\tt http://belle2.jp}
%
%\bibitem{Qt3d} {\tt https://doc.qt.io/qt-5.10/qt3d-index.html}
%
%\bibitem{Godot} {\tt https://godotengine.org/}
%
%\bibitem{Bohak2017} P. Lavrič, C. Bohak, and M. Marolt, ``Collaborative view-aligned annotations in web-based 3D medical data visualization," in MIPRO 2017: 40th Jubilee International Convention, May 22-26, 2017, Opatija, Croatia : proceedings, 2017, pp. 276-280 ({\tt http://lgm.fri.uni-lj.si/wp-content/uploads/2017/10/1537435587.pdf}).
%
%\end{thebibliography}

\sloppy
\raggedright
\clearpage
\printbibliography[title={References},heading=bibintoc]

\end{document}
